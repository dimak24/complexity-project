\documentclass[fleqn]{article}

\usepackage[T1, T2A]{fontenc}
\usepackage[utf8]{inputenc}
\usepackage[russian]{babel}
\usepackage{amsfonts} 
\usepackage{mathtools}
\usepackage{nccfoots}
\usepackage{amsthm}
\usepackage{amssymb}
\usepackage{algorithm}
\usepackage{algpseudocode}
\usepackage[right=1in, left=1in, top=1in]{geometry}
\usepackage{hyperref}
\hypersetup{
    colorlinks=true,
    linkcolor=blue,
    filecolor=magenta,      
    urlcolor=blue,
}

\theoremstyle{plain}
\newtheorem{theorem}{Теорема}

\theoremstyle{plain}
\newtheorem*{consequence*}{Следствие}

\theoremstyle{plain}
\newtheorem{lemma}{Лемма}

\theoremstyle{definition}
\newtheorem{defin}{Определение}


\author{Дмитрий Камальдинов}
\date{}
\title{Вершинная 3-раскрашиваемость графа}


\begin{document}
\maketitle

\section{Введение}
Целью данного проекта является имлементация алгоритмов проверки 3-раскрашиваемости графов.

\begin{defin}
	\emph{Правильной раскраской} графа $G = G(V,E)$ в $k$ цветов называется такое разбиение 
	$V = V_1 \sqcup V_2 \sqcup \dots \sqcup V_k$ множества его вершин на $k$ подмножеств, что 
	$\forall e \in V_i \times V_j : e \not\in E \quad (i \not= j)$.
\end{defin}

\begin{defin}
	Граф называется \emph{$k$--раскрашиваемым}, если для него существует правильная раскраска в $k$ цветов.
\end{defin}

\begin{defin} Пусть имеются множество $V$ из $n$ объектов и 
множество $C$ из $k$ \emph{цветов}. Множество цветов, в которые может быть раскрашен объект $v \in V$ обозначим через $c(v)$. Также имеется множество \emph{ограничений} $W \subset 2^{(V \times C)}$. \emph{Задача удовлетворения ограничений (CSP, constraint satisfaction problem)} заключается в отыскании такой раскраски $\alpha : V \rightarrow C$, что
\begin{itemize}
	\item $\forall v \in V \alpha(v) \in c(v)$
	\item $\forall w \in W \exists (v, c) \in w : \alpha(v) \not= c $.
\end{itemize}
\end{defin}

\begin{defin}
	(n, m)--CSP --- такая задача CSP, что
	\begin{itemize}
		\item $|C| = n$
		\item $\forall w \in W : |w| = m$
	\end{itemize}
\end{defin}

\begin{lemma}
	Задача $k$--раскрашиваемости сводится к ($k$,2)--CSP.
	\begin{proof}
		Обозначим данный граф через $G$. Следующая задача CSP эквивалентна задаче $k$--раскрашиваемости $G$:
		\begin{itemize}
			\item $V = G(V)$
			\item $C = \{1, \dots, k\}$
			\item $c(v) = C$
			\item $W = \{\{(v, c), (u, c)\} | \{v, u\} \in E(G), c \in C\}$
		\end{itemize}
	\end{proof}
\end{lemma}
\section{Описание алгоритма решения (3,2)--CSP}
Определим \emph{порядок} задачи CSP как мощность множества объектов. Объекты будут также называться \emph{вершинами}.
\begin{lemma}
	Если в задаче (3,2)--CSP порядка $n$ существует объект, который может быть покрашен в не более, чем 2 цвета, то эта задача эквивалентна некоторой порядка $n-1$.
	\begin{proof}
		Обозначим этот объект через $v$. Рассмотрим 3 случая.
		\begin{enumerate}
			\item $|c(v)| = 0$. Тогда правильной раскраски не существует, то есть задача уже решена.
			\item $|c(v)| = 1$. Обозначим единственный доступный цвет $v$ через $c$. Тогда если задача имеет решение, то $v$ обязана быть покрашеной в $c$. Значит, для каждого ограничения $\{(v, c), (u, \widehat{c})\}$, у вершины $u$ можно запретить цвет $\widehat{c}$, после чего удалить вершину $v$. Полученная задача эквивалентна исходной с учетом уже выбранного цвета для $v$, а её порядок на 1 меньше.
			\item $|c(v)| = 2$. Обозначим через $c_1, c_2$ доступные цвета для $v$. $C_i \coloneqq \{(u, c) | \{(v, c_i), (u, c)\} \in W\}$. Добавим ограничения 
$\{(u, c), (w, \widehat{c})\} \quad \forall (u,c) \in C_1, (w,\widehat{c}) \in C_2$ и удалим вершину $v$. Тогда полученная задача эквивалентна исходной. Действительно, пусть $\alpha$ --- решение (раскраска) исходной задачи. Не умаляя общности, предположим, $\alpha(v) = c_1 \implies \forall (u, c) \in C_1 : \alpha(u) \not= c$, следовательно, все новые ограничения также будут удовлетворены. Тогда $\alpha|_{V \setminus \{v\}}$ --- решение новой. Напротив, пусть $\beta$ --- решение новой задачи. Покрасим $V \setminus \{v\}$ с помощью $\beta$. Предположим противное: невозможно покрасить $v$ в один из цветов $c_1, c_2$ так, чтобы получилась правильная покраска вершин исходной задачи. Тогда существуют такие $(u, c)$ и $(w, \widehat{c})$, что $\{(u,c), (v, c_1)\} \in W$, $\{(w, \widehat{c}), (v, c_2)\} \in W$, и $\beta(u) = c$, $\beta(w) = \widehat{c}$. Но тогда $(u,c) \in C_1$ и $(w, \widehat{c}) \in C_2$, и одно из добавленных ограничений не удовлетворено.
		\end{enumerate}
	\end{proof}
\end{lemma}

\begin{lemma}
	Дана задача (3,2)--CSP порядка n, имеющая решение. Предположим, существуют два таких объекта $v$ и $u$, что $|c(v)| = |c(u)| = 3$ и $\exists c_u, c_v : \{(v, c_v), (u, c_u)\} \in W$. Тогда эта задача эквивалентна некоторой порядка $n-2$.
	\begin{proof}
		Изменим исходную задачу четырьмя способами: в каждом сохраним все ограничения, но оставим для $v$ и $u$ только по 2 возможных цвета:
		\begin{enumerate}
			\item $c(v) = \{c_v, c_v+1\}; \quad c(u) = \{c_u+1, c_u+2\}$
			\item $c(v) = \{c_v, c_v+2\}; \quad c(u) = \{c_u+1, c_u+2\}$
			\item $c(v) = \{c_v+1, c_v+2\}; \quad c(u) = \{c_u, c_u+1\}$
			\item $c(v) = \{c_v+1, c_v+2\}; \quad c(u) = \{c_u, c_u+2\}$ (везде сложение по модулю 3)
		\end{enumerate}
		Пусть одна из этих задач имеет решение $\beta$. Тогда $\beta$ также есть решение исходной задачи, так как в каждом случае невозможно покрасить $v$ и $u$ одновременно в $c_v$ и $c_u$. Напротив, пусть $\alpha$ --- решение исходной задачи, и $\alpha(v) = \widehat{c}_v, \alpha(u) = \widehat{c}_u$, где $(\widehat{c}_v, \widehat{c}_u) \not= (c_v, c_u)$. Несложным перебором можно убедиться, что в \textbf{двух}(!) из изменённых задач $c(v) \ni \widehat{c}_v$ и $c(u) \ni \widehat{c}_u$. Тогда $\alpha$ есть решение этих задач.
		
		Для завершения доказательства осталось избавиться от этих двух вершин с двумя доступными цветами, используя лемму 2.
	\end{proof}
\end{lemma}

\begin{lemma}
	Дана задача (3,2)--CSP $I$ порядка n. Cуществуют два таких объекта $v$ и $u$, что $|c(v)| = |c(u)| = 3$ и $\exists c_u, c_v : \{(v, c_v), (u, c_u)\} \in W$. Тогда её можно свести за полиномиальное время к такой $I'$, что если $I'$ разрешима, то и $I$ имеет решение, а если $I$ разрешима, то $I'$ имеет решение с вероятностью $1/2$.
	\begin{proof}
		Очевидно следует из леммы 3.
	\end{proof}
\end{lemma}

Используя лемму 4, несложно убедиться в корректности следующего алгоритма ($n$ --- количество вершин):
\begin{algorithm}[H]
  \begin{algorithmic}[1]
    \Function{satisfiable}{$I$}
    	\For {$i=1..2^{n/2}$}
			\If {$\exists v \in V : |c(v)|=3$ and $v$ has any constraints}
				\State reduce $I \rightarrow I'$ randomly using lemma 4
				\State \textbf{return} satisfiable($I'$)
			\ElsIf {$\exists v \in V : |c(v)| = 0$}
				\State \textbf{return false}
			\Else
				\State {choose any $v$ (now $0 < |c(v)| \le 2$ and reduce $I \rightarrow I'$ using lemma 2}
				\State \textbf{return} satisfible($I'$)
			\EndIf
    	\EndFor
    	\State \textbf{return false}
    \EndFunction
  \end{algorithmic}
\end{algorithm}

Действительно, если $n/2$ раз случайно выбирать вариант меньшей задачи, то, согласно лемме 4, вероятность получить правильную раскраску есть $2^{-n/2}$, то есть если вызвать алгоритм $2^{n/2}$ раз, то верный ответ будет получен с вероятностью 1. Итого время работы этого алгоритма $O(n^{O(1)}2^{n/2}) = O(1.4142135623730951^n)$


\

\begin{itemize}
	\item более подробным разбором случаев можно улучшить константу до 1.3645 (\cite{new})
	\item в случае задачи 3--раскрашиваемости константа (3,2)--CSP улучшается до 1.3289 с помощью некоторого препроцессинга на графе (см. "\emph{bushy forest}"\ в \cite{old} и \cite{new})
\end{itemize} 

\section{Эксперименты}
\begin{itemize}
	\item Алгоритм тратит на подтверждение 3--раскрашиваемости графа на 2000 вершинах до 16 секунд, в среднем 6 (плотность 0.6).
	\item До 14 секунд на нераскрашиваемых на 35 вершинах (плотность 0.5)
	\item Алгоритм работает дольше на графах с меньшей плостностью
\end{itemize}
\begin{thebibliography}{}
    \bibitem{old} Richard Beigel, David Eppstein, \emph{3--Coloring in time $O(1.3446^n)$: a no--MIS algorithm}, 36th Symposium on Foundations of Computer Science, 444--453, October 1995
    
    \url{https://ieeexplore.ieee.org/stamp/stamp.jsp?tp=&arnumber=492575}
    \bibitem{new} David Eppstein, \emph{Improved algorithms for 3--Coloring, 3--Edge--Coloring, and Constraint Satisfaction}, 12th ACM-SIAM Symp. Discrete Algorithms, Washington, 2001, pp. 32--337.
    
    \url{https://www.ics.uci.edu/~eppstein/pubs/Epp-SODA-01.pdf}
\end{thebibliography}

\end{document}